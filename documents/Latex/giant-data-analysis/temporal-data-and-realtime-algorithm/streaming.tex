\documentclass{article}

\usepackage{xeCJK}
\usepackage{indentfirst}
\bibliographystyle{alpha}

\title{现有即时海量数据处理和分析的尝试的记录}
\author{周加根(zhoujiagen@gmail.com)}
\date{2016-02-19}

% 在Mac应用>字体册.app中查看
\setCJKmainfont{Lantinghei SC Extralight}
\begin{document}

\maketitle

\renewcommand\abstractname{简述}  
\begin{abstract}
\setlength{\parskip}{0.5em} % space in paragraph

以离线的方式处理\emph{海量数据(Massive Data)}已经不符合现阶段常见的业务需求, 即时结果甚至粗糙的即时结果的获取, 对既有的数据存储、数据处理和数据分析方法提出了挑战.

本文的目的在于考察和记录现有即时海量数据处理和分析的尝试, 理清流计算的概念, 主要考察点分为流数据的分析算法和流数据处理的软件抽象.
\end{abstract}  

\newpage

\renewcommand\contentsname{目录}  
\tableofcontents  

\newpage 

\section{流计算的概念}

将流计算和批处理计算的区别讲的很清楚的一篇文章是\cite{stream-computing101} ; Akidau指出概念术语的定义, 不应该被描述成过滤是怎么使用的, 而应该采用他们是什么的本体特征定义. 基于这种考虑, 其抽象出了如下流计算相关概念: 

\begin{enumerate}
\item[(1)] 无穷数据(Unbounded data)

一种持续生成,本质上是无穷尽的数据集. 它经常会被称为“流数据”. \dots 更倾向于用无穷数据来指代无限流数据集,用有穷数据来指代有限的批次数据.

\item[(2)] 无穷数据处理(Unbounded data processing)

一种发展中的数据处理模式,应用于前面所说的无穷数据类型. \dots 只用无穷数据处理.

\item[(3)] 低延迟,近似和/或推测性结果(Low-latency,approximate,and/or speculative results)

这些结果和流处理引擎经常关联在一起. 批处理系统传统上不是设计来处理低延迟或推测性结果这个事实仅仅是一个历史产物,并无它意. 当然,如果想,批处理引擎也完全能产生近似结果. 
\end{enumerate}

将流计算系统理解为一个信息处理系统, 流数据可以视为系统录入的信息, 流数据处理则对应为系统中信息的处理, 而流数据分析可以衍生为信息本质特征的分析. 在Akidau工作的基础上, 本文从\emph{流数据}、\emph{流数据处理}和\emph{流数据分析}三个维度对流计算的概览进行讨论.

\subsection{流数据}

本节尝试分析出流数据固有的特征. 

不同的流流入信息处理系统, 这些流中数据元素的格式(schema)不必相同, 元素中一般会携带反映外部系统中事件的发生时间、事件类型标识和可能存在的事件发生现场的系统属性等信息. 

因外部系统的独立性, 流数据元素的到达速率不具备可预测性, 即在时间维度上的流数据元素的到达事件数量这一统计量不一定满足均匀分布.

常见的流数据有日志流、传感器数据、用户线上活动记录流等.

\subsection{流数据处理}

处理流数据面临的一个挑战是流中有大量的数据元素, 这一问题可以通过既有的关系型或非关系型OLTP数据库部分解决; 突发性的高速流的涌入则进一步将OLTP数据库推向了极限.

理想情况下, 流信息处理系统需要及时的识别和抽取出流中关键模式, 而不是关注流数据元素的存储非易失性, 一般不会以在线的方式立即记录流数据元素的所有属性, 否则问题就转变为实现高性能的OLTP+OLAP数据库系统问题; 然而现实中存储系统中数据写入和读取之间存在不可抗拒的矛盾.

故选用流信息处理系统的一般性假设是流中关键模式的识别和抽取(即OLAP)尽量在内存中完成, 流数据元素的存储由弱OLAP而强OLTP的数据库系统完成.

\subsection{流数据分析}

流数据本身带有时间特性, 包括所反映外部系统时间的发生时间(即事件时间)和流信息处理系统观察到或者处理流时的时间(即处理时间), 自然的流数据分析的方法需要纳入时间这一因素. 典型的应用场景有近一周内系统出现服务降级的次数、上一季度独立访问用户的数量等、今天各高峰时段用户访问量最高的资源等.

另一方面, 也需要一些时间不感知的方法, 典型的应用场景有这一地区的最高温度、服务上线以来宕机的次数等.

一个执行数据分析操作时需要权衡的问题是是否需要提供精确指标, 近似的指标是否满足需求. 在涉及资源计费时显然需要提供精确指标, 而在诸如上一季度独立访问用户的数量指标上近似的同一量级也是可以接受的.

% TODO
\newpage 
\section{流数据的分析算法}

\newpage 
\section{流数据处理的软件抽象}

\cite{storm-blueprints}
\cite{frontiers-mda}
\cite{mmd}
\cite{dataflow-model}

\newpage 
\section{总结}

\newpage
\renewcommand\refname{参考文献}  
\bibliography{references}

\end{document}


